\documentclass[times, utf8, zavrsni]{fer}
\usepackage{booktabs}

\begin{document}

% TODO: Navedite broj rada.
\thesisnumber{000}

% TODO: Navedite naslov rada.
\title{Usporedba performanci Ethereum raspodijeljene knjige na raznorodnom sklopovlju}

% TODO: Navedite vaše ime i prezime.
\author{Jakov Buratović}

\maketitle

% Ispis stranice s napomenom o umetanju izvornika rada. Uklonite naredbu \izvornik ako želite izbaciti tu stranicu.
\izvornik

% Dodavanje zahvale ili prazne stranice. Ako ne želite dodati zahvalu, naredbu ostavite radi prazne stranice.
\zahvala{Zahvaljujem se mentoru doc. dr. sc. Igoru Čavraku na podršci i pomoći pri izradi ovog rada.\\
Zahvaljujem se i kolegi Ediju Sinovčiću koji se već dulje vrijeme bavi razvojem programskih rješenja
koji koriste tehnologiju distribuirane knjige te mi je mogao odgovoriti i riješiti sve moje nedoumice.\\
}

\tableofcontents

\chapter{Uvod}
Ethereum mnogi znaju kao jednu od najpopularnijih kriptovaluta. Iako je to točno, Ethereum je mnogo
više od same digitalne valute. To je programska potpora raspodijeljena na mreži računala na kojima je
moguće razmjenjivati podatke i pokretati računalne programe bez centralnog poslužitelja. 
Podaci se repliciraju na svaki čvor u mreži. Određenim algoritmima konsenzusa se osigurava 
nepromjenjivost i pouzdanost. 
To omogućuje da mreža bude potpuno javna i sigurna. Posebnost je što svatko može koristiti
mrežu i objavljivati sadržaj bez ikakvih posebnih dozvola. \\ Za razliku od Bitcoina
i mnogih drugih implementacija tehnologije distribuirane knjige koji nude mogućnost pohrane podataka
bez centralnog poslužitelja, Ethereum mreža omogućuje i izvođenje programskog koda u obliku pametnih ugovora.\\
U ovom radu je prikazano kako se može pokrenuti privatna mreža na različitom sklopovlju s naglaskom
na ispitivanje najnižih zahtjeva sklopovlja. Dodatno, prikazat ću proces obavljanja transakcija
i izvođenja programskog koda na čvorovima uz prigodni prikaz sa sučeljem u jednostavnoj web aplikaciji.\\
U prvom poglavlju ovog rada, objasnit će se osnove tehnologije distribuirane knjige. \\
Nakon toga će biti poglavlje usmjereno na Ethereum kao implementaciju navedene tehnologije te prikaz
i opis njegovih algoritama konsenzusa. \\
Treće poglavlje je zaduženo za detaljan postupak kreiranja privatne mreže na raznorodnom sklopovlju
s različitim operativnim sustavima. \\
U posljednjem poglavlju je na prethodno kreiranu mrežu dodan grafički prikaz aktivnosti mreže i transakcija
u jednostavnoj web aplikaciji. 

\chapter{Zaključak}
Zaključak.

\bibliography{literatura}
\bibliographystyle{fer}

\begin{sazetak}
Sažetak na hrvatskom jeziku.

\kljucnerijeci{Ključne riječi, odvojene zarezima.}
\end{sazetak}

% TODO: Navedite naslov na engleskom jeziku.
\engtitle{Performance Comparison of the Ethereum Blockchain on Heterogeneous Hardware}
\begin{abstract}
Abstract.

\keywords{Keywords.}
\end{abstract}

\end{document}
